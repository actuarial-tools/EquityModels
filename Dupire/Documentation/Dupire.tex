\newcommand{\pluginName}{Dupire Local Volatility Model}
\newcommand{\pluginVersion}{1.0}

\input{../../../DocumentationTemplate/TemplateL3}

\begin{document}

\PluginTitle{\pluginName}{\pluginVersion}

\section{Introduction}
This plug-in implements the Dupire local volatility model. For a general reference on local volatility models see \cite{Dupire:PricingWSmile}, \cite{Gatheral:VolSurface}, \cite{Kamp:LocalVol}. 

\section{How to use the plug-in}

\subsection{Dupire local volatility process}

In the Fairmat user interface when you create a new stochastic process you will find the additional option ``Dupire local volatility''.

The stochastic process is defined by the parameters shown in table below:
\begin{center}
\begin{tabular}{|l|c|}
  \hline
\textbf{Fairmat}&\textbf{Documentation}\\
\textbf{notation}&\textbf{notation}\\
                     \hline
 S0 	& $S_0$\\
 Time Dependent Risk Free Rate (Zero Rate) & $r(t)$\\
 Time Dependent Continuous Dividend Yield & $q(t)$\\
 Local Volatility & $\sigma(t, s)$\\
   \hline
\end{tabular}
\end{center}
$S_0$ is the starting values for the stock process $r(t)$ and $q(t)$ are one dimensional function while $\sigma(t, s)$ is a two dimensional function.

Remember that to specify a reference to a previous defined function {\ttfamily f} you have to use the notation {\ttfamily @f}.

\section{Implementation Details}

The Dupire local volatility model is used to describe the evolution of a stock price (or an index) with a volatility that is a function of time and index value. The process is defined by the following stochastic differential equation

\begin{equation}
dS(t) = (r(t)-q(t)) S(t)dt + \sigma(t,S(t))S(t)dW(t)\label{eq:sde}\\
\end{equation}
where $S$ represents the price process, and $dW$ is a Wiener processes.

\subsection{Simulation and discretization scheme}

Applying straight Euler-Maruyama method to simulate a local volatility process, we obtain this formula
\begin{equation}
S_{n+1} = S_n + (r(t_n) - q(t_n))S_n\Delta t + \sigma(t_n,S_n)\sqrt{\Delta t} N(0,1)
\end{equation}
where $\Delta t = t_{n+1}-t_n$ and $N(0,1)$ represents a realization of a standard normal random variable. With this kind of discretization it is possible for $S$ to reach negative values.

So the best approach is that of simulate $log(S(t))$ process. By Ito's lemma from equation (\ref{eq:sde}) we can deduce
\begin{equation}
d\left[\log\left(S(t)\right)\right] = \left(r(t) - q(t)- \frac{1}{2}\sigma^2(t, S(t)) \right)dt + \sigma(t, S(t))dW % CONTROLLARE!!!
\end{equation}
and its discrete counterpart is
\begin{align}
\log(S_{n+1}) = \log(S_n) + \left(r(t_n) - q(t_n) - \frac{1}{2}\sigma^2(t_n, S_n) \right)\Delta t\nonumber\\
 + \sigma(t_n, S_n)\sqrt{\Delta t} N_1(0,1)
\end{align}
Simulating the stock price this way entails no discretization error and solve the problem of generating negative index value.

\section{Calibration}

It can be demonstrated that supposing to have a continuous surface of implied volatilities $\Sigma(t,S)$ the surface of local volatility is complitely determined by Dupire's formula
\begin{equation}\label{eq:LocVolFromImpVol}
\sigma^2(t, S) = \frac{ \Sigma^2 + 2\Sigma t (\frac{\partial\Sigma}{\partial t} + (r(t) - q(t))S\frac{\partial\Sigma}{\partial S} ) }{ \left( 1- \frac{yS}{\Sigma} \frac{\partial\Sigma}{\partial S} \right)^2 + tS\Sigma\left( \frac{\partial\Sigma}{\partial S} - \frac{1}{4} tS\Sigma \left(\frac{\partial\Sigma}{\partial S}\right)^2 + S\frac{\partial^2\Sigma}{\partial S^2} \right) }
\end{equation}
where for simplicity we suppressed the $(t,S)$ dependency of $\Sigma$ and where
\begin{equation}
y(t,S) = \ln\left( \frac{S}{S_0} \right) + \int_{0}^t(q(s) - r(s))ds
\end{equation}

Of course the main problem with this model is that the implied volatilitly surface $\Sigma$ is not given, but the market provides us only an implied volatility matrix.

The first step is then to decide how to interpolate the implied volatility matrix to give a smooth surface on which calculate derivatives that appear in formula (\ref{eq:LocVolFromImpVol}). This can be done in several ways...



\bibliographystyle{unsrt}
\bibliography{../../../DocumentationTemplate/bibliography}
\end{document}


